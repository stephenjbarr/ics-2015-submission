%%%%%%%%%%%%%%%%%%%%%%%%%%%%%%%%%%%%%%%%%%%%%%%%%%%%%%%%%%%%%%%%%%
% file: ics-2015-paper-sample.tex
% for authors of proceedings accompanying ICS 2015
% author: Mirko Janc
% last update:  May 8, 2014
%%%%%%%%%%%%%%%%%%%%%%%%%%%%%%%%%%%%%%%%%%%%%%%%%%%%%%%%%%%%%%%%%%
%%%%%%%%%%%%%%%%%%%%%%%%%%%%%%
\documentclass{ics-2015}
%%%%%%%%%%%%%%%%%%%%%%%%%%%%%%

%%%%%%%%%%%%%%%%
\begin{document}
%%%%%%%%%%%%%%%%

\setcounter{page}{1} %

\YEAR{2015}%
\FIRSTPAGE{000}%
\LASTPAGE{000}%
\DOI{http://dx.doi.org/10.1287/ics.2015.XXXX}

%\CHAPTERNO{}%   Not in use

\TITLE{Papers Accompanying the 14th ICS Conference---Style Instructions}

%%\SUBTITLE{Subtitle Should Generally Be Avoided}% comment out if not necessary

\AUBLOCK{% Numbers next to the \AUTHOR are just for reference---they do not appear
%
\AUTHOR{Mirko Janc}
\AFF{INFORMS, 5521 Research Park Drive, Catonsville, Maryland 21228, USA,
\EMAIL{mirko.janc@informs.org}}
%
\AUTHOR{Second User, Third Author}
\AFF{Computer Science Department, University of Anystate, Someville, Anystate 02005, USA,
\textbraceleft\EMAIL{second.user@anystate.edu}, \EMAIL{third.author@anystate.edu}\textbraceright}
%
}

\RUNNINGHEAD{Janc, User, and Author: ICS-2015 Style Instructions}

\ABSTRACT{The text here should be your abstract. If at all possible, avoid having more than
one paragraph. Formulas are not forbidden but should be avoided because the abstract will be
also ``out there'' in Internet environments that are not ``math-friendly.'' References from
the abstract should also be avoided. If a reference is absolutely necessary, it should be
shown in full. Under the abstract are the keywords. They should be listed in their natural
case, as a semicolon-separated list. The text that follows gives additional explanations of
how to use the style for producing papers for the proceedings, to be published 
open access.}

\KEYWORDS{ICS 2015; operations research publications; paper
preprints; LaTeX templates; LaTeX styles}

\maketitle

%%%%%%%%%%%%%%%%%%%%%%%%%%%%%%%%%%%%%%%%%%%%%%%%%%%%%%%%%
\section{Introduction}%1
%%%%%%%%%%%%%%%%%%%%%%%%%%%%%%%%%%%%%%%%%%%%%%%%%%%%%%%%%

We give a brief overview of some elements that appear in the {\ICS} style. The style is based
on standard fonts that come with TeX for free.
%The usual LaTeX font loading process is somewhat altered to reduce the number of PostScript Type1 fonts used.
It is assumed that you run LaTeX2e (not the obsolete LaTeX 2.09).
Your system should use PostScript Type1 Computer Modern fonts (that are truly scalable!),
not the bitmapped fonts.
Type1 fonts come with any of currently available popular TeX systems---free systems (MiKTeX, teTeX, OzTeX, etc.),
or commercial TeX implementations (PCTeX, YYTeX, Textures, etc.).

For versatility of handling mathematics, \verb+amsmath+ and
\verb+amssymb+ packages are preloaded by the style---you do not have to
declare them via \verb+\usepackage+. For inclusion of graphics,
\verb+graphicx+ package is also automatically preloaded by the style.
Because of the everywhere present Web addresses, package \verb+url+ is also preloaded.
For LaTeX related questions that are not a part of ``how to use this style?'',
please refer to any of LaTeX textbooks or references.

One often neglected element of style is the treatment of upper- and lowercase in a sentence
or a fragment. Our style does not allow all uppercase titles/heads. The other two
possibilities and their usage in the {\ICS} style are shown in Table~\ref{st-style}.


%%%%%%%%%%%%%%%%%%%%%%%
\begin{table}
\TABLE
{\textit{Title style} vs.\ \textit{sentence style}: overview of the usage.\label{st-style}}
{\begin{tabular*}{\hsize}{@{\extracolsep{\fill}}p{1.2in}p{1.5in}p{2.3in}@{}}
\hline
\up\down Case style& \mc{Example}& \mc{Usage}\\
\hline
\RG\up Title style %%%\newline (aka CLC = caps and lowercase)
& \RG Example Bibliography & \RG Chapter title; section, subsection,
subsubsection, paragraph, and subparagraph titles; running heads;
heads of all theorem-like environments; book, journal, dissertation, etc. titles in references\\[4pt]
\RG Sentence style %%%(aka ICLC = initial cap and (then) lowercase)
& \RG Example bibliography & \RG Figure and table captions; figure and
table notes; table column heads; words and sentence fragments as table body entries;
article titles in references\down\\
\hline
\end{tabular*}}
{}
\end{table}

\subsection{Section Numbering}% 1.1

Sections are numbered in the usual
way. Generally, the style allows for numbering down to level 5, which is the
standard LaTeX term for the sequence of headings \verb+\section+,
\verb+\subsection+, \verb+\subsubsection+, \verb+\paragraph+, and
\verb+\subparagraph+. Depending on the structural complexity of your chapter,
use the command \verb+\setcounter{secnumdepth}{<depth>}+ to set an appropriate level of numbering.

Excessive numbering reduces the clarity of exposition. Heads should be numbered only as deep
as it is necessary. Typically, values
\verb+...{secnumdepth}{3}+ or
\verb+...{secnumdepth}{2}+ are good choices. The style default is ``\verb+3+''. Do not change it unless you
have good reasons to do so.

\subsection{Equation, Theorem, Figure, and Table Numbering}% 1.2

As a rule of thumb, each heading you use should fit into one
of the sectioning categories (numbered or not) or into one of the
theorem-like environments (theorems, lemmas, propositions, corollaries,
claims, definitions, remarks, proofs, etc.).

Other than section numbering, two other major groups of counting
sequences are equations and theorem-like environments (enunciations).
The way they are numbered is determined by the style.
If you have a side comment with any of the theorem-like environments' titles, declare it
in the optional argument

\begin{verbatim}
\begin{lemma}[Farkas, see \cite{bookwebDynamic}]
Text of the Farkas lemma.
\end{lemma}
\end{verbatim}
which gives

\begin{lemma}[Farkas, see \cite{bookwebDynamic}]
Text of the Farkas lemma.
\end{lemma}

You will notice that some theorem-like environments are set in italic
and separated from the surrounding text by some space, whereas the
others are all-roman and have no extra space above and below. The style
takes care of all such issues. Do not manually space out any of the
enunciations.

For {\ICS} chapters our style prefers (and enforces) simple numbering of figures and tables.
Both figures and tables will be numbered as 1, 2,~$\ldots.$ Captions will start with
\textsc{Figure~1}, \textsc{Figure 2}, \textsc{Table 1}, \textsc{Table 5}, etc.


%%%%%%%%%%%%%%%%%%%%%%%%%%%%%%%%%%%%%%%%%%%%%%%%%%%%%
\section{Figures and Tables}
%%%%%%%%%%%%%%%%%%%%%%%%%%%%%%%%%%%%%%%%%%%%%%%%%%%%%

For handling figures and tables, we provide macros \verb+\FIGURE+
and \verb+\TABLE+ to capture the content of a floating element (figure or table), its caption, and possible notes.
Captions should be kept simple, typically as a sentence fragment. Any additional information, including
table footnotes, should be set as the third argument to
\verb+\FIGURE+ or and \verb+\TABLE+.


\begin{figure}[t]
\FIGURE
{A figure spanning the whole width of the text.\label{fig1}}
{\framebox{\hbox to 388bp{\HD{47}{30}\hss\fs.24.30. A\ \ \ W I D E\ \ \ F I G U R E\hss}}}
{We use type instead of a real figure to reduce the size of this file.}
\end{figure}

A typical case is Figure~\ref{fig1}:

{\fs.8.9.\relax
\begin{verbatim}
\begin{figure}[t]
\FIGURE
{A figure spanning the whole width of the text.\label{fig1}}
{\framebox{\hbox to 388bp{\HD{47}{30}\hss\fs.24.30. A\ \ \ W I D E\ \ \ F I G U R E\hss}}}
{We use type instead of a real figure to reduce the size of this file.}
\end{figure}
\end{verbatim}}

\noindent Note that \verb+\FIGURE+ has three arguments:
(1)~caption with label,
(2)~included graphics that will be automatically centered in the available \verb+\hsize+,
and
(3)~optional (foot)note. If you do not need
the note field just leave it as an empty pair of braces \verb+{}+. For syntax, it must be
present even in that (empty) case.

For inclusion of graphics, use the \verb+\includegraphics+ syntax from \verb+graphicx+. For example,
if the figure name is \verb+fracimage1.eps+ and it should span the full width of the text (5.5~in),
you would include
\begin{verbatim}
\includegraphics[width=5.5in]{fracimage1.eps}
\end{verbatim}
as the second argument to \verb+\FIGURE+.


\begin{figure}[t]
\begin{minipage}[t]{160bp}
\FIGURE
{A small figure.\label{fig2}}
{\framebox{\hbox to 153bp{\HD{37}{30}\hss\TEN A SMALL FIGURE\hss}}}
{See \texttt{www.asmallfigure.com}.}
\end{minipage}
\hfill
\begin{minipage}[t]{218bp}
\FIGURE
{A not so small figure.\label{fig3}}
{\framebox{\hbox to 211bp{\HD{37}{30}\hss\TEN A NOT SO SMALL FIGURE\hss}}}
{A figure note here.}
\end{minipage}
\end{figure}


Similar macro, \verb+\TABLE+, has three arguments:
(1)~caption with label,
(2)~table itself by using the produced by \verb+tabular+ or \verb+tabular*+ environments,
and
(3)~optional (foot)note.
Per style, tables have only three horizontal lines: above
the table, between table column heads and the table body, and after the
table body. Vertical rules are discouraged.

If the table is too big, it may be necessary to reduce its type size.
Such font (and other) adjustments
should be done within the second argument of \verb+\TABLE+, just before
\verb+\begin{tabular*}+.


\begin{table}[b]
{\begin{minipage}{2in}
\TABLE
{A little caption.\label{tab1}}
{\begin{tabular*}{2in}{@{}l@{\extracolsep{\fill}}l@{}}
\hline
\up State& Capital\down\\
\hline
\up Maryland& Annapolis\\
Virginia& Richmond\\
Colorado& Denver\down\\
\hline
\end{tabular*}}
{A note?}
\end{minipage}}%
\hfill
{\begin{minipage}{3in} \TABLE {A somewhat longer caption
just to get into the second line.\label{tab2}}
{\begin{tabular*}{3in}{@{\extracolsep{\fill}}ll@{\hskip16pt}ll@{}}
\hline
\up State& Capital& State& Capital\down\\
\hline
\up Maryland& Annapolis& Pennsylvania& Harrisburg\\
Virginia& Richmond& Georgia &Atlanta\\
Colorado& Denver& Illinois& Springfield\down\\
\hline
\end{tabular*}}
{}
\end{minipage}}
\end{table}



To make the table better styled regarding vertical spacing around rules,
use macro \verb+\up+ in one of the entries after the horizontal lines
and \verb+\down+ in one of the entries before horizontal lines. The small Table~\ref{tab1}
is coded as

{\fs.8.9.\relax
\begin{verbatim}
\TABLE
{A little caption.\label{tab1}}
{\begin{tabular*}{2in}{@{}l@{\extracolsep{\fill}}l@{}}
\hline
\up State& Capital\down\\ \hline
\up Maryland& Annapolis\\
Virginia& Richmond\\
Colorado& Denver\down\\ \hline
\end{tabular*}}
{A note?}
\end{verbatim}}


%%%%%%%%%%%%%%%%%%%%%%%%%%%%%%%%%%%%%%%%%%%%%
\section{Math Formulas}
%%%%%%%%%%%%%%%%%%%%%%%%%%%%%%%%%%%%%%%%%%%%%

This style preloads \verb+amsmath+ and \verb+amssymb+. That means you
can use both the classic LaTeX formatting tools for equations as
\verb+equation+ and \verb+eqnarray+, and the variety of \verb+amsmath+ supplied tools.
Special letters as script and openface (blackboard bold) should be coded as, for example,
\verb+\mathcal{R},\mathbb{R}+ to get $\mathcal{R}, \mathbb{R}$. Avoid obsolete and ugly constructions as
$\textrm{I{\kern-0.17em}R}$ for openface letters.

In the \verb+array+ environment in math, entries are by default set in
text style (smaller version of sum, product, union, integral, etc. is used,
limits of summation set as subscripts and superscripts, not below and above, etc.).
Fractions by default set as \verb+\tfrac+ (small text fractions),
not as \verb+\dfrac+ (larger display-style fractions).
Occasionally that looks too cramped.
To fix the problem, precede those fields by \verb+\DS+ (for \verb+\displaystyle+). To improve vertical
spacing in such cases, \verb+\mcr+ may be used instead of the usual
\verb+\\+. Compare
\begin{equation}\label{eq2}
\left\lbrace \begin{array}{l@{\quad}l}
x\sin\frac1x,&x\ne0\\ 0,&x=0\end{array} \right.
\quad\text{or}\quad
\left\lbrace \begin{array}{l@{\quad}l}
\DS x\sin\frac1x,&x\ne0\\ 0,&x=0\end{array} \right.
\quad\text{with}\quad
\left\lbrace \begin{array}{l@{\quad}l}
\DS x\sin\frac1x,&x\ne0\mcr 0,&x=0\end{array} \right.
\end{equation}
{\fs.8.9.\relax
\begin{verbatim}
\left\lbrace \begin{array}{l@{\quad}l}
x\sin\frac1x,&x\ne0\\ 0,&x=0\end{array} \right.
\quad\text{or}\quad
\left\lbrace \begin{array}{l@{\quad}l}
\DS x\sin\frac1x,&x\ne0\\ 0,&x=0\end{array} \right.
\quad\text{with}\quad
\left\lbrace \begin{array}{l@{\quad}l}
\DS x\sin\frac1x,&x\ne0\mcr 0,&x=0\end{array} \right.
\end{verbatim}}

Another typical problem arises when you need a new math
operator, as for example ``support,'' in $\mathop{\textrm{supp}}(K)$.
Define it as
\begin{verbatim}
\def\supp{\mathop{\textrm{supp}}\nolimits}
\end{verbatim}
If your operator needs limits as in
\begin{equation}\label{eq3}
2\pi i\,\mathop{\textrm{Res}}_{z=\pi i} f(z),
\end{equation}
define it as
\begin{verbatim}
\def\Res{\mathop{\textrm{Res}}}
\end{verbatim}
so that the above formula will be typed as \verb+2\pi i\Res_{z=\pi i}f(z)+.

%%%%%%%%%%%%%%
\begin{table}[t]
\TABLE {All the different types of bibliography entries 
\label{tabtut}} {\begin{tabular*}{\hsize}{@{}l@{\extracolsep\fill}l@{}} \hline
\up\down Type of entry&\multicolumn{1}{c}{Comments}\\
\hline
\up
   article  & \parbox[t]{4.39in}{published \cite{articlepublished};
             to appear \cite{articletoappear}; \\
             submitted with journal named \cite{articlesubmittedjournalNamed}; \\
             submitted without journal named \cite{articlesubmittedjournalNotNamed}}\\[25pt]
%\hline
   book     & regular \cite{book};
             web, static \cite{bookwebStatic};
             web, dynamic \cite{bookwebDynamic} \\[4pt]
%\hline
   database & with author \cite{databasewithAuthor} ;
              without author \cite{databasewithoutAuthor} \\[4pt]
%\hline
   in book  & \parbox[t]{4.39in}{Chapter \cite{chapter}
              \dots book is in series \cite{chapterinseries};
              in proceedings \cite{inproceedings}}\\[4pt]
%\hline
   proceedings & \cite{proceedings} \\[4pt]
%\hline
   manual   & with author \cite{manualwithAuthor};
              without author (also dynamic) \cite{manualwithoutAuthor};
              primer \cite{primer} \\[4pt]
%\hline
   newsletter & \parbox[t]{4.39in}{format same as article;
                here is one with added note of online availability and
                double publication \cite{newsletter}} \\[14.5pt]
%\hline
   proceedings & \cite{proceedings} \\[4pt]
%\hline
   report  & \cite{reportacademic, reportgovt} \\[4pt]
%\hline
   thesis  & Ph.D. \cite{thesisPHD}; Master's \cite{thesisMS} \\[4pt]
%\hline
   web site & no author \cite{website}; course \cite{websitecourse};
             item \cite{websiteitem}\down \\
\hline
\end{tabular*}}
{This table and accompanying explanations were provided by Harvey Greenberg \cite{Greenberg04}.}
\end{table}
%%%%%%%%%%%%%%



%%%%%%%%%%%%%%%%%%%%%%%%%%%%%%%%%%%%%%%%%%%%%%%%%%%%%
\section{Example Bibliography}
%%%%%%%%%%%%%%%%%%%%%%%%%%%%%%%%%%%%%%%%%%%%%%%%%%%%%

Table \ref{tabtut} illustrates different kinds of documents you could reference in your
proceedings paper.  If you use BibTeX, you merely specify plain style (but you must still use
some of the optional fields and not abbreviate names of journals and proceedings). Otherwise,
please note the formats.  In particular, all authors begin with their first (and other)
initials, followed by their last name.

You might see that some entries have a note after the citation, such as
``Available on\-line~...'' \cite{manualwithAuthor} and
``This is a course site \dots'' \cite{websitecourse}.  You are free to do an
annotated bibliography.  If you use BibTeX, the \verb+TutORials.bst+ style file
supports an \texttt{annote} field.

%%%%%%%%%%%%%%%%%%%%%%%%%%%%%%%%%%%%%%%%%%%%%%%%%%%%%
\begin{APPENDIX}{}%{With a Title}
%%%%%%%%%%%%%%%%%%%%%%%%%%%%%%%%%%%%%%%%%%%%%%%%%%%%%

\noindent If you want an appendix with a title, please fill
in the title into the empty braces at the end of \verb+\begin{APPENDIX}{}+.
Appendix sections will be numbered by A, B, C, etc.

\section{My First Appendix Section}

The text goes on. The text goes on. The text goes on. The text goes on. The text goes on. The text goes on.
The text goes on. The text goes on. The text goes on. The text goes on. The text goes on. The text goes on.

\subsection{First Appendix Subsection}

In this subsection, there will be even a theorem and a lemma.
Their numbering continues from the main text. The same holds for numbering of equations.

\begin{theorem}
If this is a theorem in your appendix, for any real numbers $a,b,c$ the equality
\begin{equation}
a^3+b^3= (a+b)(a^2-ab+b^2)
\end{equation}
holds, regardless of the smaller text size in which this statement is presented.
\end{theorem}

Here is also a lemma.

\begin{lemma}
This is an appendix lemma.
\end{lemma}

\subsubsection{An Example of Appendix Subsubsection.}

Ignore this text. It is just to fill in a paragraph.
Ignore this text. It is just to fill in a paragraph.
Ignore this text. It is just to fill in a paragraph.
Ignore this text. It is just to fill in a paragraph.

\section{Second Appendix Section}

Ignore this text too. It is just to fill in a paragraph.
Ignore this text too. It is just to fill in a paragraph.
Ignore this text too. It is just to fill in a paragraph.
Ignore this text too. It is just to fill in a paragraph.

\subsection{Second Appendix Section's Subsection}

This is the final section shown. After this we finish the appendix
by issuing the command \verb+\end{APPENDIX}+ and go down to the references.

\end{APPENDIX}

%\section{References}

%%\bibliographystyle{TutORials} % Use this if you need annotations
\bibliographystyle{plain}
\bibliography{procbiblio}

%%%%%%%%%%%%%%%%
\end{document}
%%%%%%%%%%%%%%%%

%%%%%%%%%%%%%%%%%%%%%%%%%%%%%%%%%%%%%%%%%%%%%%%%%%%%%%%%%%%%%%%%%%
% file: ICS-2015-paper-template.tex
% for authors of INFORMS Computing Society Conference (ICS-2015)
% author: Mirko Janc
% last update:  May 7, 2014
%%%%%%%%%%%%%%%%%%%%%%%%%%%%%%%%%%%%%%%%%%%%%%%%%%%%%%%%%%%%%%%%%%
%%%%%%%%%%%%%%%%%%%%%%%%%
\documentclass{ics-2015} % put ./ics-2015 if using only locally
%%%%%%%%%%%%%%%%%%%%%%%%%

%%%%%%%%%%%%%%%%%%%%%%%%%%%%%%
% Packages that are automatically loaded by ics-2015 style file:
%   amsmath, amssymb
%   graphicx
%   url
% (documentation at www.ctan.org).
%%%%%%%%%%%%%%%%%%%%%%%%%%%%%

%%%%%%%%%%%%%%%%
\begin{document}
%%%%%%%%%%%%%%%%

\setcounter{page}{1} %

\YEAR{2015}%
\FIRSTPAGE{000}%
\LASTPAGE{000}%
\DOI{http://dx.doi.org/10.1287/ics.2015.XXXX}

\TITLE{A Functional Programming Approach to Dynamic Programs Arising in Inventory Management}    % Enter chapter title here

%%\SUBTITLE{}% Enter substitle (only if enecessary) and outcomment

% Enter author(s) here:
\AUBLOCK{%
  \AUTHOR{Stephen J. Barr and Hamed Mamani}
  \AFF{Information Systems and Operations Management, University of Washington, Seattle, WA 98195, % \EMAIL is a part of \AFF(iliation)
       \EMAIL{stevejb@uw.edu}}
  \AFF{Information Systems and Operations Management, University of Washington, Seattle, WA 98195, % \EMAIL is a part of \AFF(iliation)
       \EMAIL{hmamani@uw.edu}}
  % ... up to last author
} % This ends \AUBLOCK

\RUNNINGHEAD{}% Shortened running head "Author(s): Short Title"

\ABSTRACT{%
% put abstract here (required)

Finding an optimal policy in an inventory setting often involves formulating and solving dynamic programs.
These types of problems are notoriously slow to solve, and tractability concerns limit exploration of the solution space.
In the computing realm, functional programming (FP) has enjoyed a surge of innovation in recent years, motivated pragmatically by its role in parallel computing and theoretically by a rich category-theoretic underpinning.
Additionally, FP compilers can implement aggressive optimizations due to the lack of mutable state.

We pose a problem consisting of a manufacturer and multiple retailers, in which one retailer has access to additional information about the manufacturer.
We formulate this as a finite horizon dynamic program, and implement a solver in Haskell, a leading FP language.
Our implementation achieves drastic speed increases relative to our reference Matlab implementation.
This allows us to explore the solution space over a broad range of parameterizations of the model.
Additionally, we demonstrate a simulated method of moments (SMM) estimation, which matches moments of hypothetical data set with moments resulting from a simulated optimal policy.
Our approach is applicable to a broad range of finite-horizon dynamic programs.

} % This ends the abstract

\KEYWORDS{%
functional programming; inventory; dynamic programming;
} % This ends the keywords

%%%%%%%%%%%
\maketitle  % This will make the chapter opening ("titlepage")
%%%%%%%%%%%

% \section{Introduction} % Make the first section and introduction

% body of the paper -- number objects normally, as with article style;
%                      see ics-2011-paper-sample for a sample submission
%                      and additional explanations

% \section{Summary}    % name your last section whatever you like

% Appendix optional
%\APPENDIX{This is the Appendix Title}
%\APPENDIX{}  % <-- This is an appendix without a title.

%\section{References}
% Make bibliography with BibTeX (or type out if necessary)
% Use
%   plain.bst
% or, if annotations are needed,
%   TutORials.bst

\bibliographystyle{plain}
%\bibliographystyle{TutORials}   % put ./TutORials.bst if using locally
\bibliography{procbiblio}  % put your bib file name here

%%%%%%%%%%%%%%%%
\end{document}
%%%%%%%%%%%%%%%%

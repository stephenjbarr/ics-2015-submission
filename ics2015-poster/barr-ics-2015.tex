\documentclass{tikzposter} % See Section 3

\usepackage{listings,relsize}

\usepackage{natbib}
\usepackage{minted}
\usepackage{boiboites}

\usepackage{tcolorbox}
\usepackage{etoolbox}
\BeforeBeginEnvironment{minted}{\begin{tcolorbox}}%
\AfterEndEnvironment{minted}{\end{tcolorbox}}%


\makeatletter
\def\title#1{\gdef\@title{\scalebox{\TP@titletextscale}{%
\begin{minipage}[t]{\linewidth}
\centering
#1
\par
\vspace{0.5em}
\end{minipage}%
}}}
\makeatother

\title{A Functional Programming Approach to \\ Dynamic Programming Problems in Inventory Management} \institute{Foster School of Business, University of Washington } % See Section 4.1
\author{Stephen J. Barr} % \titlegraphic{Logo}
\usetheme{Autumn} % See Section 5
\begin{document}
\maketitle % See Section 4.1

\begin{columns}
\column{0.5}
\block{Finite Horizon Dynamic Program}{
  \begin{itemize}
  \item $N$ period solution horizon
  \item \textbf{Objective: (Inventory)} Find optimal stock levels for retailer 
  \item Solve using backwards induction
  \end{itemize}

  Notation:
  \begin{equation}
    \label{eq:dp-recursion}
    x_{k+1} = f_{k}(x_{k}, u_{k}, w_{k}), k \in [0,...,N-1]
  \end{equation}
  where
  \begin{itemize}
  \item $x_{k}$ system state
  \item $u_{k}$ control variable
  \item $w_{k}$ random disturbance
  \end{itemize}

} % See Section 4.2

\column{0.5}
\block{Why Functional Programming?}{Haskell}
\end{columns}

\begin{columns} % See Section 4.4
\column{0.3} % See Section 4.4
\block{Imperative Pseudo-Code}{

\inputminted{c}{calg.c}

\textbf{Advantages:}
\begin{itemize}
\item Easily recognizable
\item Familiar
\item Seems to clearly translate Eq. \ref{eq:dp-recursion}
\end{itemize}

}

\column{0.3}
\block{Notes About Imperative}{
  \begin{itemize}
  \item Solution at each iteration is a matrix $\mathbf{Y}^{t}$.
  \item Full solution is the set of matrices $[\mathbf{Y}^{1},...,\mathbf{Y}^{K}]$
  \item Every element $\mathbf{Y}^{k}_{i,j} = f(Y^{k+1}; \Theta)$.
  \end{itemize}
}
% \note{Notetext} % See Section 4.3
\end{columns}
\end{document}
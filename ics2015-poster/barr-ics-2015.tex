\documentclass{tikzposter} % See Section 3

\usepackage{listings,relsize}

\usepackage{natbib}
\usepackage{bibentry}
\usepackage{minted}
\usepackage{boiboites}
\usepackage{amsmath}
\usepackage{tcolorbox}
\usepackage{etoolbox}
\usepackage{booktabs}
\BeforeBeginEnvironment{minted}{\begin{tcolorbox}}%
\AfterEndEnvironment{minted}{\end{tcolorbox}}%


\makeatletter
\def\title#1{\gdef\@title{\scalebox{\TP@titletextscale}{%
\begin{minipage}[t]{\linewidth}
\centering
#1
\par
\vspace{0.5em}
\end{minipage}%
}}}
\makeatother

\title{A Functional Programming Approach to \\ Dynamic Programming Problems in Inventory Management} \institute{Foster School of Business, University of Washington } % See Section 4.1
\author{Stephen J. Barr} % \titlegraphic{Logo}
\usetheme{Autumn} % See Section 5
\begin{document}
\maketitle % See Section 4.1

\begin{columns}
\column{0.5}
\block{Finite Horizon Dynamic Program}{
  \begin{itemize}
  \item $N$ period solution horizon
  \item \textbf{Objective: (Inventory)} Find optimal stock levels for retailer 
  \item Solve using backwards induction
  \end{itemize}

  Notation:
  \begin{equation}
    \label{eq:dp-recursion}
    x_{k+1} = f_{k}(x_{k}, u_{k}, w_{k}), k \in [0,...,N-1]
  \end{equation}
  where
  \begin{itemize}
  \item $x_{k}$ system state
  \item $u_{k}$ control variable
  \item $w_{k}$ random disturbance
  \end{itemize}

} % See Section 4.2

\column{0.5}
\block{Why Functional Programming?}{Haskell}
\end{columns}

\begin{columns} % See Section 4.4
\column{0.3} % See Section 4.4
\block{Imperative Pseudo-Code}{

\inputminted{c}{calg.c}

\textbf{Advantages:}
\begin{itemize}
\item Easily recognizable
\item Familiar
\item Seems to clearly translate Eq. \ref{eq:dp-recursion}
\end{itemize}


\textbf{Notes:}
\begin{itemize}
  \item Solution at each iteration is a matrix $\mathbf{Y}^{t}$.
  \item Full solution is the set of matrices $[\mathbf{Y}^{1},...,\mathbf{Y}^{K}]$
  \item Every element $\mathbf{Y}^{k}_{i,j} = f(Y^{k+1}; \Theta)$.
  \end{itemize}


}

% \column{0.3}
% \block{Notes About Imperative}{
% \textbf{Imperative Notes:}
%   \begin{itemize}
%   \item Solution at each iteration is a matrix $\mathbf{Y}^{t}$.
%   \item Full solution is the set of matrices $[\mathbf{Y}^{1},...,\mathbf{Y}^{K}]$
%   \item Every element $\mathbf{Y}^{k}_{i,j} = f(Y^{k+1}; \Theta)$.
%   \end{itemize}

% }

\column{0.7}
\block{How this often gets implemented}{
  Mathworks\textregistered Matlab / GNU Octave Style
  \inputminted{octave}{matlabcode1.m}

  \textbf{Observation:} This style imposes far more structure on the problem than what is often intended and needed to solve the problem.
}

\end{columns}


\begin{columns}
  \column{0.5}
  \block{Implicit Limitations of Imperative Approach}{
    \begin{enumerate}
    \item The nested for loops suggest an ordering. 
      For a given period $k$, calculations are performed in the order $[(1,1),(1,2),...,(1,i_{max}),(2,1)...]$. This is \textbf{overspecifying} the solution, and thus limiting the compiler's ability to optimize.
    \end{enumerate}
  }
  \column{0.5}
  \block{Functional Equivalent}{ 
    \begin{enumerate}
    \item Create a set over indices $\mathcal{I}$ and map over this set. This specifies what we want to happen, without limiting the compiler to an exection strategy.
    \end{enumerate}
  }


\end{columns}

\begin{columns}
  \column{0.4}
  \block{map in Haskell}{
    \inputminted{haskell}{simplemap.hs}
  
    \begin{itemize}
    \item Imposes exactly the amount of required structure.
    \item Compiler is free to optimize the execution including, (CPU parallelism [1], usage of GPU [2], or both [3])
    \item With minimal effort, you can run on a cluster [4].
    \end{itemize}

  }

\column{0.6}
\block{Scan}{
  The backward induction loop is equivalent to a scan. 

      \inputminted{haskell}{scanl.hs}              
  
  % In the type signature, the function \mintinline{haskell}{f} has type  \mintinline{haskell}{a -> b -> a}. Let type \mintinline{haskell}{a} be a matrix type, and \mintinline{haskell}{b} be an integer type. The, function \mintinline{haskell}{f} given to \mintinline{haskell}{scanl}, for this application, takes a matrix, and integer, and returns a matrix.

  \vspace{1.0cm}

  \begin{center}
    \begin{tabular}{ l | l | l }
      \toprule 
      Expression & Our Application & Math \\
      \hline
      \mintinline{haskell}{f :: a -> b -> a} & f :: Matrix -> Int -> Matrix   & $ \mathbf{Y}_{k} = f(\mathbf{Y}_{k+1}, k+1)  $  \\
      \mintinline{haskell}{z :: a} & z :: Matrix & Null initial matrix \\
      \mintinline{haskell}{[x] :: b} & b :: [Int] & $[N,N-1,...,1]$ \\
      \hline
    \end{tabular}
  \end{center}


}



\end{columns}

% \note{Notetext} % See Section 4.3

\begin{columns}
  \column{0.33}
  \block{Notation}{

    Solution vector 
    \begin{align*}
      &\equiv [\mathbf{Y}_{N}, \mathbf{Y}_{N-1}, ..., \mathbf{Y}_{1}] \\
      % &= [g(\mathbf{0}, N), g(g(\mathbf{0}, N-1)), ..., g(...(g(\mathbf{0},1))...)]
      &= [g(-, N), g(g(-, N-1)), ..., g(...(g(-,1))...)] \\
      &= g^{\circ k}(-;\Theta) ~ k \in \{N,N-1,...1\}
    \end{align*}
    % where
    % \begin{equation*}
    %   g(\mathbf{X}, k) =
    %   \begin{cases}
    %     f'(\Theta)  & k = N \\
    %     f(\mathbf{X}, k; \Theta) & k \in [N-1] 
    %   \end{cases}
    % \end{equation*}

    % \begin{itemize}
    % \item     $g(k ) = 
    %   \begin{cases}
    %     f'(\Theta) &k = N \\
    %     f^{\circ k}(\Theta) &k \in \{N-1,...,1\}
    %   \item  $[\mathbf{Y}_{N}, \mathbf{Y}_{N-1}, ..., \mathbf{Y}_{1}]$
    %   \item $ \equiv [f'(\Theta), f(f'(\Theta)), f(f(f'(\Theta))), ..., f^{\circ N}(x)]$.
    %   \end{itemize}

  }


  \column{0.34}
  \block{SMM}{

  }

  \column{0.33}
  \block{Useful packages}{
    \begin{itemize}
    \item [1] Par monad - \citet*{marlow_monad_2011}  
    \item [2] Accelerate - \citet*{mcdonell_optimizing_2014}
    \item [3] Meta Par - \citet*{foltzer_meta-scheduler_2012}
    \item [4] Cloud Haskell -  \citet*{epstein_towards_2011}
    \item [5] LLVM backend - \citet*{terei_llvm_2010}
    \end{itemize}
  }


\end{columns}

  
\nobibliography{/home/stevejb/Dropbox/bibliography/sjbmainbibtex}
\bibliographystyle{plainnat}


\end{document}